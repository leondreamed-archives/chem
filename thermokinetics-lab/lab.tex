\documentclass[12pt, notitlepage, letterpaper]{report}
\usepackage[version=4]{mhchem}
\usepackage{siunitx}
\usepackage[margin=1in]{geometry}
\usepackage{tabularx}
\usepackage{ragged2e}
\usepackage[font=small, skip=3pt]{caption}
\usepackage{calculator}
\usepackage{float}
\usepackage{graphicx}

\DeclareSIUnit\ml{\milli\litre}
\DeclareSIUnit\mpl{\mol\per\litre}
\DeclareSIUnit\mmol{\milli\mole}
\DeclareSIUnit\gpl{\gram\per\mol}
\DeclareSIUnit\kjpmol{\kilo\joule\per\mol}
\DeclareSIUnit\heatcap{\joule\per\celsius\per\gram}

\sisetup{space-before-unit = true, free-standing-units = true}
\graphicspath{{assets/}}

\title{Thermokinetics Lab}
\author{Leon Si}
\date{February 22, 2022}

\begin{document}
\maketitle

% Rubric:

% Sufficient relevant quantitative and qualitative raw data supports a detailed and valid research question conclusion.
% Appropriate and sufficient data processing is carried out with the accuracy required to validate conclusion; fully consistent with the experimental data.
% Full and appropriate consideration of the impact of measurement uncertainty on the analysis.
% Processed data correctly interpreted so that a completely valid and detailed conclusion to the research question can be deduced.

% Presentation of the investigation is clear. Errors do not hamper understanding of the focus, process and outcomes.
% Clear report structure; necessary information on focus, process and outcomes is present and presented coherently.
% Report is relevant and concise; facilitates a ready understanding of the focus, process and outcomes.
% Subject specific terminology and conventions is appropriate and correct. Errors do not hamper understanding.

\section*{Data}

% Declare data
% \begin{noindent}
<?
/*
[0]: Volume of CuSO4 Time of Zn Addition
[1]: Initial temp
[2]: Time of Zn addition
[3]: b value for equation of cooling line
[4]: m value for equation of cooling line (could be positive/negative)
[5]: R value of cooling line
*/

const rawData = [
	// Eric, Bobby, Bryan	20	24.4	72	75.358 - 0.0417x	-0.9918
	[20, 24.4, 72, 75.358, -0.0417, -0.9918],
	// Estelle, Hannah, Molly 	20	24.2	132	76.057 - 0.0521x	-0.9591
	[20, 24.2, 132, 76.1, -0.0521, -0.9591],
	// Lily, Cindy, Devanshi	20	24.2	60	74.909-0.086299x	-0.99338
	[20, 24.2, 60, 74.9, -0.086299, -0.99338],
	// Kevin, Leon, Thomas	20	24.8	72	38.948 - 0.0052021x	-0.95144
	[20, 24.8, 72, 38.948, -0.0052021, -0.95144],
	// Avaneesh, Artin, Ethan, Yucen	20	24.6	96	75.883 - 2.5313x	-0.99564
	[20, 24.6, 96, 75.883, -2.5313 / 60, -0.99564], // time of cooling line was given in minutes
	// Colin, Henry, Collin, Ishan	20	23.1	14	35.707+0.0060035x	0.952365476
	[20, 23.1, 14, 35.707, +0.0060035, 0.952365476],
	// Jasmine, Amy, Tina	20	23.8	15	74.526-0.059536x	-0.8973
	[20, 23.8, 15, 74.526, -0.059536, -0.8973],
	// Prachi, Michelle, Naina 	20	24.6	96	61.881 - 0.02918	-0.99694
	[20, 24.6, 96, 61.881, -0.02918, -0.99694],
]

const keys = ['volume', 'initialTemp', 'zincAdditionTime', 'coolingLineB', 'coolingLineM', 'rValue'];

// The Ramda library (https://ramdajs.com/) is present in the global scope as `R`!
const data = rawData.map((row) => Object.fromEntries(R.zip(keys, row)));

function getRowEquation(rowIndex) {
	const row = data[rowIndex];
	const sign = row.coolingLineM > 0 ? '+' : '-';
	const equation = `${sigfig(row.coolingLineM, 3)} ${sign} ${sigfig(Math.abs(row.coolingLineM), 3)}$x$`;
	return equation;
}
?>
% \end{noindent}

\begin{table}[H]
	\caption{Data collected from 8 trials of measuring the heat released from the reaction between \ce{CuSO4} and \ce{Zn}. The equation of the cooling line approximates the temperature of the solution in degrees Celsius at $x$ seconds, and the parameters are rounded to 3 significant figures.}
	\def\arraystretch{1.5}
	\begin{tabularx}{\linewidth}{|
			>{\RaggedRight}X|
			>{\RaggedRight}X|
			>{\RaggedRight}X|
			>{\RaggedRight}X|
			>{\RaggedRight}X|
			>{\RaggedRight}X|
		}
		\hline
		Trial \#
		 & Volume of \ce{CuSO4} /\ml
		 & Baseline temperature /\celsius
		 & Time of \ce{Zn} addition /\second
		 & Equation of cooling line
		 & R value
		\\\hline
		% \begin{noindent}
		<? for (const [rowIndex, row] of data.entries()) { ?>
			<?= rowIndex + 1 ?>
			& <?= row.volume ?>
			& <?= row.initialTemp ?>
			& <?= row.zincAdditionTime ?>
			& <?= getRowEquation(rowIndex) ?>
			& <?= sigfig(row.rValue, 3) ?>
			\\\hline
		<? } ?>
		% \end{noindent}
	\end{tabularx}
\end{table}

\section*{Analysis}

The data in Table 1 can be used to find the molar enthalpy of formation of \ce{ZnSO4}. For the following calculations, the values in Trial 4 are used.

To find the molar enthalpy of formation of \ce{ZnSO4}, the moles of \ce{ZnSO4} ($n_{ZnSO4}$) and the energy change of the reaction $Q_{reaction}$ must be known.

\subsection*{Moles of \ce{ZnSO4}}

To find the moles of \ce{ZnSO4} produced, the moles of \ce{CuSO4} must be knonwn. Using the known concentration of \ce{CuSO4} (1\mpl), the volume of \ce{CuSO4} can be converted to moles of \ce{CuSO4}:
\begin{align*}
	c     & = \frac{n}{V}        \\
	1\mpl & = \frac{n}{20\ml}    \\
	n     & = 20\mmol~\ce{CuSO4}
\end{align*}

Then, to find the moles of \ce{ZnSO4}, the balanced chemical equation of the reaction between zinc metal (\ce{Zn}) and aqueous copper sulfate (\ce{CuSO4}) is used:

\centerline{\ce{CuSO4_{(aq)} + Zn_{(s)} -> ZnSO4_{(aq)} + Cu_{(s)}}}

This equation gives the mole ratio between \ce{CuSO4} and \ce{ZnSO4}, which can be used to find the moles of \ce{ZnSO4} produced in the reaction:
\begin{align*}
	n_{\ce{ZnSO4}} & = 20\mmol~\ce{CuSO4} * \frac{1\mol~{\ce{ZnSO4}}}{1\mol~{\ce{CuSO4}}} = 20\mmol~{\ce{ZnSO4}}
\end{align*}

\subsection*{Change in energy of the reaction ($Q_{reaction}$)}

To calculate the molar enthalpy of reaction between \ce{CuSO4} and \ce{Zn}, the change in energy of the reaction ($Q_{reaction}$) must also be known. The energy lost by the reaction is equal to the energy gained by the solution: $Q_{reaction} = -Q_{solution}$. To find $Q_{solution}$, the mass $m$ of the solution, the specific heat capacity $c$ of the solution, and the temperature change $\Delta T$ of the solution are all needed.

Since the solution is dilute, the mass of the solution can be approximated using the density of water, 1~\unit{\gram\per\cm}. Thus, the mass of the solution is:
\begin{align*}
	m & = \rho * V                      \\
	m & = 1~\unit{\gram\per\cm} * 20\ml \\
	m & = 20\gram
\end{align*}

The heat capacity of the solution is approximated with the specific heat capacity of water (4.184\heatcap). Thus, $c = 4.184\heatcap$

To calculate the change in temperature of the solution, the initial and final temperatures are needed. The initial temperature of the solution before the reaction ($T_i$) is equal to the \textit{Baseline temperature} column in Table 1.

To obtain the final temperature of the solution, the cooling of the solution needs to be taken into account. This is done by plugging in the time when \ce{Zn} was added into the solution into the equation of the cooling line.
\begin{align*}
	T_f & = 38.948 - 0.0052021x    \\
	T_f & = 38.948 - 0.0052021(72) \\
	T_f & = 38.5734488\celsius
\end{align*}

This value represents the point $x$ in the following graph:

\begin{figure}[h!]
	\caption{A graph of temperature (\unit{\celsius}) against time (\unit{\second}) of the reaction between \ce{CuSO4} and \ce{Zn}. The orange line represents the line of best fit of the cooling line, the red line indicates the time when zinc was added, and the red circle indicates the final temperature of the solution with cooling taken into account.}
	\centerline{\noindent\includegraphics[width=0.8\textwidth]{time-vs-temperature-marked.png}}
\end{figure}

With these values, the energy change of the solution and the energy change of the reaction can be determined:
\begin{align*}
	Q_{solution} & = mc\Delta T = mc(T_f - T_i)
	\\
	Q_{solution} & = (20\gram) * (4.184\heatcap) * (38.5734\celsius - 24.8\celsius)
	\\
	Q_{solution} & = 1152.56\joule
	\\
	Q_{reaction} & = -Q_{solution}
	\\
	Q_{reaction} & = -1152.56\joule
\end{align*}

\subsection*{Molar enthalpy of reaction ($\Delta H_{reaction}$)}

Using the moles of \ce{ZnSO4} and the change in energy of the reaction ($Q_{reaction}$), the molar enthalpy of the reaction can be calculated:
%
\begin{align*}
	\Delta H_{reaction} & = \frac{Q_{reaction}}{n}                        \\
	\Delta H_{reaction} & = \frac{-1152.56\joule}{0.02\mol}               \\
	\Delta H_{reaction} & = -57628~\unit{\joule\per\mol} = -57.628\kjpmol
\end{align*}

Performing the above calculation for all trials in Table 1 yields the following table of values:

% \begin{noindent}
<?
function getFinalTemp(rowIndex) {
	const row = data[rowIndex];
	const tFinal = row.coolingLineM * row.zincAdditionTime + row.coolingLineB;
	return tFinal;
}

function getQReaction(rowIndex) {
	const row = data[rowIndex];
	const m = row.volume;
	const c = 4.184;
	const tInitial = row.initialTemp;
	const tFinal = getFinalTemp(rowIndex);
	const q = -1 * m * c * (tFinal - tInitial);
	return q;
}

function getHReaction(rowIndex) {
	const q = getQReaction(rowIndex);
	const mols = 0.02;
	const h = q / mols / 1000; // kJ/mol
	return h;
}
?>

\begin{table}[H]
	\caption{Calculations of $\Delta H_{reaction}$ for all trials.}
	\def\arraystretch{1.5}
	\begin{tabularx}{\linewidth}{|
			p{0.1\linewidth}|
			p{0.3\linewidth}|
			>{\RaggedRight}X|
			>{\RaggedRight}X|
		}
		\hline
		Trial \#
		 & Final Temperature $T_f$ /\unit{\celsius}
		 & $Q_{reaction}$ /\unit{\joule}
		 & $\Delta H_{reaction}$ /\unit{\kjpmol}
		\\\hline
		% \begin{noindent}
		<? for (const [rowIndex, row] of data.entries()) { ?>
			<?= rowIndex + 1 ?>
			& <?= sigfig(getFinalTemp(rowIndex), 3) ?>
			& <?= sigfig(getQReaction(rowIndex), 3) ?>
			& <?= sigfig(getHReaction(rowIndex), 3) ?>
			\\\hline
		<? } ?>
		% \end{noindent}
\end{tabularx}
\end{table}

The experimental value for $\Delta H_{reaction}$ can be determined by taking the average of the values from all 8 trials. The uncertainty can be calculated using the half-range method:
\vspace{5pt}

<?
const hReactions = Array.from({ length: data.length }).map((_, i) => getHReaction(i));
const avgHReaction = R.mean(hReactions);
const avgHReactionRounded = sigfig(avgHReaction, 3);
?>
%
\begin{align*}
	\Delta H_{reaction} & = \frac{-201 + -188 + -191 + -57.6 + -198 + -53.1 + -209 + -144}{8} \\
	\Delta H_{reaction} & = (<?= avgHReactionRounded ?> \pm 80)\kjpmol
\end{align*}

\subsection*{Comparison with Theoretical Value}
To find the theoretical value of this reaction, the following formula is used:
%
\begin{align*}
	\Delta H_{reaction} = \sum \Delta {H_f}_{products} - \sum \Delta {H_f}_{reactants}
\end{align*}
%
According to NIST, the molar enthalpy formation of \ce{CuSO4} is -769.98\kjpmol and the molar enthalpy of formation of \ce{ZnSO4} is -980.14\kjpmol . Because \ce{Zn} and \ce{Cu} are in their standard states, the molar enthalpy of \ce{Zn} and \ce{Cu} is 0\kjpmol .
\begin{align*}
	\Delta H_{reaction} & = -980.14\kjpmol - (-769.98\kjpmol) \\
	\Delta H_{reaction} & = -210.16\kjpmol
\end{align*}

This theoretical value can be used to determine the percent error of the value obtained from the experimental trials:
%
<?
const expectedHReaction = -980.14 - (-769.98);
const percentError = Math.abs((avgHReaction - expectedHReaction) / expectedHReaction);
?>
%
\begin{align*}
	\delta & = |\frac{v_A - v_E}{v_E}| * 100\%                                  \\
	\delta & = |\frac{<?= avgHReactionRounded ?> - (-210.16)}{-210.16}| * 100\% \\
	\delta & = <?= sigfig(percentError * 100, 3) ?>\%
\end{align*}

However, from the experimental data, there are potentially two outliers: Trial 3 and Trial 5. Because there is potentially more than one outlier, the Dixon Q test cannot be used since it is only valid for data that has exactly one outlier. To determine these outliers in our data, the Tietjen-Moore test is used instead.

The Tietjen-Moore test calculates a test statistic $L_k$ based on the data. The value of the test statistic is between zero and one: if there are no outliers in the data, the test statistic is very close to 1, and if there are outliers in the data, it will be closer to 0. For a data set with exactly $k$ outliers, the formula for calculating the test statistic $L_k$ for a Tietjen-Moore test is:
\begin{align*}
	L_k = \frac{\sum_{i=1}^{n-k}(y_i-\bar{y}_k)^2}{\sum_{i=1}^{n}(y_i-\bar{y})^2}
\end{align*}

where $n$ is the number of data points in our data set, $\bar{y}$ denotes the mean for all trials, and $\bar{y}_k$ denotes the sample mean with the largest $k$ points removed. It is suspected that there are 2 outliers in the experimental results displayed in Table 2. Thus, $k = 2$:
%
<?
// Retrieving the data with the two largest values removed
const filteredHReactions = hReactions.sort((a, b) => a - b).slice(0, -2);
const fullMean = R.mean(hReactions);
const filteredMean = R.mean(filteredHReactions);
const numerator = R.sum(filteredHReactions.map(y => (y - filteredMean) ** 2));
const denominator = R.sum(hReactions.map(y => (y - fullMean) ** 2));
const L_2 = numerator / denominator;
?>
%
\begin{align*}
	L_2 & = \frac{\sum_{i=1}^{8-2}(y_i-\bar{y}_2)^2}{\sum_{i=1}^{8}(y_i-\bar{y})^2} \\
	L_2 & = <?= sigfig(L_2, 3) ?>
\end{align*}

Because this test statistic is close to zero, it strongly suggests that there are two outliers in the data. Removing these outliers (Trial 3 and Trial 5) from our data and recalculating the $\Delta H_{reaction}$ gives a much smaller percent error:
%
<?
const avgHReaction2 = R.mean(filteredHReactions);
const avgHReaction2Rounded = sigfig(avgHReaction2, 3);
const percentError2 = Math.abs((avgHReaction2 - expectedHReaction) / expectedHReaction);
?>
%
\begin{align*}
	\Delta {H_{reaction}}_{2} & = \frac{-201 + -188 + -191 + -198 + -209 + -144}{6}                 \\
	\Delta {H_{reaction}}_{2} & = (<?= avgHReaction2Rounded ?> \pm 30)\kjpmol                       \\
	\delta                    & = |\frac{<?= avgHReaction2Rounded ?> - (-210.16)}{-210.16}| * 100\% \\
	\delta                    & = <?= sigfig(percentError2 * 100, 3) ?>\%
\end{align*}

\section*{Conclusion}

The experimental value of the molar enthalpy of reaction between \ce{CuSO4} and \ce{Zn} was calculated to be -188.31\kjpmol . Because this number is positive, it indicates that the reaction is exothemic and releases energy as heat. Compared to the theoretical value of -210.16\kjpmol , the experimental value is greater, indicating that the reaction did not release as much energy as expected.

This discrepancy can be explained by the assumptions made in our calculations. We assumed that all the heat from the reaction is being transferred to the solution through our equation $Q_{reaction} = -Q_{solution}$. However, during this experiment, some of this heat was also transferred to the container: the styrofoam cup, and our calculations did not take into account the heat capacity of the container. As a result, the energy gained by the solution is less than the energy released by the reaction, leading to a molar enthalpy calculation that is lower than the theoretical molar enthalpy.

Another potential source of error is the extrapolation of the solution's final temperature based on the cooling line. The R values in Table 1 indicate that there is some uncertainty when approximating the equation of the cooling line, leading to a discrepancy between the extrapolated final temperature and the actual final temperature after the zinc was added. A possible explanation for this is an unfinished reaction between \ce{CuSO4} and \ce{Zn}, which causes the temperature to fluctuate some time after the zinc was added. This effect is evident in Figure 1, where the temperature rises between 350s and 375s, and especially after 500s which is when the solution was stirred more vigorously, indicating that the copper sulfate and zinc reaction hadn't gone to completion.

\end{document}
