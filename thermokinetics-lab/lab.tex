\documentclass[12pt, notitlepage, letterpaper]{report}
\usepackage[utf8]{inputenc}
\usepackage[version=4]{mhchem}
\usepackage{siunitx}
\usepackage[margin=1in]{geometry}
\usepackage{tabularx}
\usepackage{ragged2e}
\usepackage{caption}

\DeclareSIUnit\ml{\milli\litre}
\DeclareSIUnit\mpl{\mol\per\litre}
\DeclareSIUnit\mmol{\milli\mole}
\DeclareSIUnit\gpl{\gram\per\mol}
\DeclareSIUnit\kjpmol{\kilo\joule\per\mol}
\DeclareSIUnit\heatcap{\joule\per\celsius\per\gram}

\sisetup{space-before-unit = true, free-standing-units = true}

\title{Thermokinetics Lab}
\author{Leon Si}
\date{February 22, 2022}

\begin{document}
\maketitle
% Sufficient relevant quantitative and qualitative raw data supports a detailed and valid research question conclusion.
% Appropriate and sufficient data processing is carried out with the accuracy required to validate conclusion; fully consistent with the experimental data.
% Full and appropriate consideration of the impact of measurement uncertainty on the analysis.
% Processed data correctly interpreted so that a completely valid and detailed conclusion to the research question can be deduced.

\section*{Data}

\begin{table}[hbt!]
	\caption{Data collected from 8 trials of the experiment.}
	\def\arraystretch{1.5}
	\begin{tabularx}{\linewidth}{|
			>{\RaggedRight}X|
			>{\RaggedRight}X|
			>{\RaggedRight}X|
			>{\RaggedRight}X|
			>{\RaggedRight}X|
			>{\RaggedRight}X|
		}
		\hline
		Trial \#
		 & Volume of \ce{CuSO4} /\ml
		 & Baseline temperature /\celsius
		 & Time of \ce{Zn} addition /\second
		 & Equation of cooling line
		 & R value
		\\\hline
		% Eric, Bobby, Bryan	20	24.4	72	75.358 - 0.0417x	-0.9918
		1
		 & 20
		 & 24.4
		 & 72
		 & 75.358 - 0.0417x
		 & -0.992
		\\\hline
		% Estelle, Hannah, Molly 	20	24.2	132	76.057 - 0.0521x	-0.9591
		2
		 & 20
		 & 24.2
		 & 132
		 & 76.057 - 0.0521x
		 & -0.959
		\\\hline
		% Lily, Cindy, Devanshi	20	24.2	60	74.909-0.086299x	-0.99338
		3
		 & 20
		 & 24.2
		 & 60
		 & 74.909 - 0.086299x
		 & -0.993
		\\\hline
		% Kevin, Leon, Thomas	20	24.8	72	38.948 - 0.0052021x	-0.95144
		4
		 & 20
		 & 24.8
		 & 72
		 & 38.948 - 0.0052021x
		 & -0.951
		\\\hline
		% Avaneesh, Artin, Ethan, Yucen	20	24.6	96	75.883 - 2.5313x	-0.99564
		5
		 & 20
		 & 24.6
		 & 96
		 & 75.883 - 2.5313x
		 & -0.996
		% Colin, Henry, Collin, Ishan	20	23.1	14	35.707+0.0060035x	0.952365476
		\\\hline
		6
		 & 20
		 & 23.1
		 & 14
		 & 35.707 + 0.0060035x
		 & 0.952
		\\\hline
		% Jasmine, Amy, Tina	20	23.8	15	74.526-0.059536x	-0.8973
		7
		 & 20
		 & 23.8
		 & 15
		 & 74.526 - 0.059536x
		 & -0.897
		\\\hline
		% Prachi, Michelle, Naina 	20	24.6	96	61.881 - 0.02918	-0.99694
		8
		 & 20
		 & 24.6
		 & 96
		 & 61.881 - 0.02918x
		 & -0.997
		\\\hline
	\end{tabularx}
\end{table}

\section*{Analysis}

To find the molar enthalpy of formation of \ce{ZnSO4}, the energy change of the reaction $Q_{reaction}$ and the moles of \ce{ZnSO4} ($n_{ZnSO4}$) must be known.

To find the moles of \ce{ZnSO4} produced, the moles of \ce{CuSO4} must be knonwn. Using the known concentration of \ce{CuSO4} (1\mpl), the volume of \ce{CuSO4} can be converted to moles of \ce{CuSO4}:
\begin{align*}
	c     & = \frac{n}{V}     \\
	1\mpl & = \frac{n}{20\ml} \\
	n     & = 20\mmol
\end{align*}

Then, to find the moles of \ce{ZnSO4}, the balanced chemical equation of the reaction between zinc metal (\ce{Zn}) and aqueous copper sulfate (\ce{CuSO4}) is used:

\centerline{\ce{CuSO4_{(aq)} + Zn_{(s)} -> ZnSO4_{(aq)} + Cu_{(s)}}}

This equation gives the mole ratio between \ce{CuSO4} and \ce{ZnSO4}, which can be used to find the moles of \ce{ZnSO4} produced in the reaction:
\begin{align*}
	n_{\ce{ZnSO4}} & = 20\mmol\,\ce{CuSO4} * \frac{1\mol\,{\ce{ZnSO4}}}{1\mol\,{\ce{CuSO4}}} = 20\mmol\,{\ce{ZnSO4}}
\end{align*}

In addition to the number of moles of \ce{ZnSO4}, the change in energy of the reaction ($Q_{reaction}$) must also be known. The energy lost by the reaction is equal to the energy gained by the solution: $Q_{reaction} = -Q_{solution}$. To find $Q_{solution}$, the mass $m$ of the solution, the specific heat capacity $c$ of the solution, and the temperature change $\Delta T$ of the solution are all needed.

Since the solution is dilute, the mass of the solution can be approximated using the density of water, 1~\unit{\gram\per\cm}. Thus, the mass of the solution is:
\begin{align*}
	m & = \rho * V                       \\
	m & = 1\,\unit{\gram\per\cm} * 10\ml \\
	m & = 10\gram
\end{align*}

The heat capacity of the solution is approximated with the specific heat capacity of water (4.184\heatcap). Thus, $c = 4.184\heatcap$

To calculate the change in temperature of the solution, the initial and final temperatures are needed. The initial temperature of the solution before the reaction ($T_i$) can be found in Table 1 under \textit{Baseline temperature}: 24.8\celsius.

To obtain the final temperature of the solution, the cooling of the solution needs to be taken into account. This is done by plugging in the time when \ce{Zn} was added into the solution into the equation of the cooling line (Table 1):
\begin{align*}
	T_f & = 38.948 - 0.0052021x    \\
	T_f & = 38.948 - 0.0052021(72) \\
	T_f & = 38.5734488\celsius
\end{align*}

With these values, the energy change of the solution and the energy change of the reaction can be determined:
\begin{align*}
	Q_{solution} & = mc\Delta T = mc(T_f - T_i)
	\\
	Q_{solution} & = (10\gram) * (4.184\heatcap) * (38.5734\celsius - 24.8\celsius)
	\\
	Q_{solution} & = 576.28\joule
	\\
	Q_{reaction} & = -Q_{solution}
	\\
	Q_{reaction} & = -576.28\joule
\end{align*}

Finally, the molar enthalpy of the reaction can be calculated:
\begin{align*}
	\Delta H_{reaction} & = \frac{Q_{reaction}}{n}                         \\
	\Delta H_{reaction} & = \frac{-576.28\joule}{0.02\mol}                 \\
	\Delta H_{reaction} & = -28814\,\unit{\joule\per\mol} = -28.814\kjpmol
\end{align*}

\subsection*{Comparison with Theoretical Value}
To find the theoretical value of this reaction, the following formula is used:

\begin{align*}
	\Delta H_{reaction} = \sum \Delta {H_f}_{products} - \sum \Delta {H_f}_{reactants}
\end{align*}

According to NIST, the molar enthalpy formation of \ce{CuSO4} is -769.98\kjpmol and the molar enthalpy of formation of \ce{ZnSO4} is -980.14\kjpmol . Because \ce{Zn} and \ce{Cu} are in their standard states, the molar enthalpy of \ce{Zn} and \ce{Cu} is 0\kjpmol .
\begin{align*}
	\Delta H_{reaction} & = -980.14\kjpmol - (-769.98\kjpmol) \\
	\Delta H_{reaction} & = -210.16\kjpmol
\end{align*}

% Presentation of the investigation is clear. Errors do not hamper understanding of the focus, process and outcomes.
% Clear report structure; necessary information on focus, process and outcomes is present and presented coherently.
% Report is relevant and concise; facilitates a ready understanding of the focus, process and outcomes.
% Subject specific terminology and conventions is appropriate and correct. Errors do not hamper understanding.
\end{document}