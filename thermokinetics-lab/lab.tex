\documentclass[12pt, notitlepage, letterpaper]{report}
\usepackage[utf8]{inputenc}
\usepackage[version=4]{mhchem}
\usepackage{siunitx}
\usepackage[margin=1in]{geometry}
\usepackage{tabularx}
\usepackage{ragged2e}

\DeclareSIUnit\ml{\milli\litre}
\sisetup{space-before-unit = true, free-standing-units = true}

\title{Thermokinetic e Lab}
\author{Leon Si}
\date{February 22, 2022}

\begin{document}
\maketitle
% Sufficient relevant quantitative and qualitative raw data supports a detailed and valid research question conclusion.
% Appropriate and sufficient data processing is carried out with the accuracy required to validate conclusion; fully consistent with the experimental data.
% Full and appropriate consideration of the impact of measurement uncertainty on the analysis.
% Processed data correctly interpreted so that a completely valid and detailed conclusion to the research question can be deduced.

\def\arraystretch{1.5}
\begin{tabularx}{\linewidth}{|
		>{\RaggedRight}X|
		>{\RaggedRight}X|
		>{\RaggedRight}X|
		>{\RaggedRight}X|
		>{\RaggedRight}X|
	}
	\hline
	Volume of \ce{CuSO4} /\ml
	 & Baseline temperature /\celsius
	 & Time of \ce{Zn} addition /\second
	 & Equation of cooling line
	 & R value
	\\\hline
	20
	 & 24.8
	 & 72
	 & 38.948 - 0.0052021x
	 & -0.95144
	\\\hline
\end{tabularx}


% Presentation of the investigation is clear. Errors do not hamper understanding of the focus, process and outcomes.
% Clear report structure; necessary information on focus, process and outcomes is present and presented coherently.
% Report is relevant and concise; facilitates a ready understanding of the focus, process and outcomes.
% Subject specific terminology and conventions is appropriate and correct. Errors do not hamper understanding.
\end{document}